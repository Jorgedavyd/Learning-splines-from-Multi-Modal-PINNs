\documentclass[12pt]{article}

\usepackage{amsmath}
\usepackage{amsfonts}
\usepackage{graphicx}
\usepackage{hyperref}
\usepackage{cite}

\title{Physics informed Learnable Splines for Zero-shot Satellite Data Generation}
\author{Jorge D. Enciso}
\date{\today}

\begin{document}

\maketitle

\begin{abstract}
\end{abstract}

\tableofcontents
\newpage

\section{Introduction}

\section{Related Work}

\subsection{Physics informed Spline Learning for Nonlinear Dynamics Discovery}

\subsection{Neural Operator: Learning Maps Between Function Spaces}

\subsection{Spline-PINN: Approaching PDEs without Data Using Fast, Physics-Informed Hermite-Spline CNN}

\section{Physics Informed Machine Learning}
Usually, solely data-driven methods require huge volumes of high-quality data with black box algorithms that prevent interpretability, a key factor for physical modeling. The most recent research \cite{PINNS} advocate for physical informed virtual loss function terms, leading off to the vastly known field of Phyics Informed Neural Networks.

This models where created as numerical methods for solving Partial Differential Equations that embed physical modeling. It further extended to all fields that required PDE solving as a computationally efficient numerical alternative to other solvers.

The most recent research, using PINNs, develop a way to not just learn functions, but to learn functional operators: Lagrangian, Hamiltonian, etc. This research \cite{} creates a neural network that takes the general coordinates $q$ and $\dot q$ as input, to provide the lagrangian as an output, and leverage the auto-differentiation capabilities of modern machine learning frameworks to enforce the euler-lagrange equation:

\begin{equation}
    \frac{d}{dt} \frac{\partial \mathcal{L}}{\partial \dot q} - \frac{\partial \mathcal{L}}{\partial q} = 0
\end{equation}

As usage of $q$ and $\dot q$ as inputs is very convenient for the auto-differentiation engines, making it further computationally feasible.

\section{Data}

\subsection{DSCOVR: Deep Space Climate Observatory}
DSCOVR, a joint mission between NASA and the National Oceanic and Atmospheric Administration, is a crucial observational platform for monitoring space weather \cite{nasa_dscovr}. Launched in 2015, DSCOVR's primary mission is to monitor and provide advanced warning of potentially hazardous space weather events such as solar flares and coronal mass ejections that could impact Earth.

It is equipped with two key instruments for measuring both energetic particle incidence and magnetic field parameters: the Faraday cup and the magnetometer from the PlasMag instrument \cite{nasa_dscovr}. The readings from these two sensors are crucial for virtually analyzing plasma dynamics near the L1 Lagrange point. These readings will be used as part of the core model data due to their real-time availability.

\subsection{ACE: Advanced Composition Explorer}
ACE, launched in 1997, provides continuous measurements of the solar wind and interstellar particles. It is equipped with several instruments designed to study the composition of solar and galactic particles, which are crucial for understanding the space weather environment. ACE's data helps in predicting geomagnetic storms and contributes to our understanding of the heliosphere.

\subsection{WIND}
The WIND spacecraft, launched in 1994, is part of the Global Geospace Science initiative. It provides comprehensive measurements of the solar wind, magnetic fields, and energetic particles. WIND's data is essential for understanding the fundamental processes of the solar wind and its interaction with the Earth's magnetosphere.

\subsection{SOHO: Solar and Heliospheric Observatory}
SOHO, a joint project of ESA and NASA, was launched in 1995. It is designed to study the Sun from its core to the outer corona and the solar wind.

\subsubsection{LASCO: Large Angle and Spectrometric Coronagraph Observatory}
LASCO, one of the instruments on SOHO, observes the solar corona by creating an artificial eclipse. It is instrumental in detecting coronal mass ejections, which are significant drivers of space weather.

\section{Methods}

\subsection{Neural Architecture Search (NAS)}
As an intermediate step before training, a hyperparameter grid search algorithm based in Bayesian multi-objective optimization is employed to find the neural architecture. The Pareto front is plotted, selecting minimization of neural complexity and loss function.

\subsection{Loss function}


\section{Results}

\section{Discussion}


\section{Conclusion}
This work demonstrated the effectiveness of Physics-Informed Neural Operators as splines to improve zero-shot satellite data generation. This behavior was tied to diverse empirical and analytical formulations. By embedding the governing physical equations directly into the loss function, we can efficiently train a universal approximators to acquire a representational power over important measuremnts of our universe.

\bibliographystyle{plain}

\bibliography{references}
\end{document}
